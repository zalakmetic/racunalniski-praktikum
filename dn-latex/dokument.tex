\documentclass[11pt]{article}
\usepackage[a4paper, margin=2.5cm]{geometry}
\usepackage[slovene]{babel}
\usepackage[utf8]{inputenc}
\usepackage[T1]{fontenc}
\usepackage{amsmath}
\usepackage{amsthm}
\usepackage{graphicx}
\usepackage{amsfonts}

\theoremstyle{definition}
\newtheorem{definicija}{Definicija}

\theoremstyle{plain}
\newtheorem{izrek}{Izrek}

\newcommand{\f}{\mathcal{F}}

\title{Brownovo gibanje}
\author{Matej Rojec}
\date{}

\begin{document}
 
\maketitle

Brownovo gibanje (več v !!) je intuitivno slučajen proces, \cite{karatzas_shreve}
ki predstavlja naključno gibanje delcev v mediju.
    
    \begin{figure}[h]
        \centering
        \includegraphics[width=0.7\textwidth]{PerrinPlot2.pdf}
        \caption{Reprodukcija slike iz Jean Baptite Perrin, \emph{Mouvement brownien et réalité moléculaire}, Ann. de Chimie et de Physique (VIII) 18, 5-114, 1909}
    \end{figure}
    % Slika: PerrinPlot2.pdf
    % Napis pod sliko: 
    % Reprodukcija slike iz Jean Baptiste Perrin, \emph{Mouvement brownien et réalité moléculaire}, Ann. de Chimie et de Physique (VIII) 18, 5-114, 1909

    
    \begin{definicija}
        Standardno Brownovo gibanje $\{B_t\}_{t \geq 0}$ je slučajen proces z naslednjimi lastnostmi: 
            \begin{enumerate}
                \item $B_0 = 0$.
                \item Prirastki $B_{t_n} - B_{t_{n-1}}, B_{t_{n-1}} - B_{t_{n-2}}, \ldots, B_2 - B_1, B_1 - B_0$ so neodvisne slučajne spremenljivke, za vsak $t_0 \leq t_1 \leq \cdots \leq t_{n-1} \leq t_n$.
                \item Za vsak $t \geq 0$ in $h > 0$ velja $B_{t+h} - B_t \sim \mathcal{N}(0, h)$.
                \item Funkcija $t \mapsto B_t$ je zvezna skoraj gotovo.
            \end{enumerate}
    \end{definicija}
    
    
    Preden zapišemo izrek, definirajmo še pojem časa ustavljanja.
    
  
    \begin{definicija}
    Slučajna spremenljivka $\tau$ na verjetnostnem prostoru \((\Omega, \f, P)\) z vrednostmi v \(\mathbb{R}^+\)
    je čas ustavljanja glede na filtracijo \((\f_t)_{t \in T}\), če velja \(\f_tau\).
    \end{definicija}
    
    Zdaj lahko zapišemo izrek \ref{thm:stopped_brownian}.
    
   
    \begin{izrek}\label{thm:stopped_brownian}
    Naj bo $\{B_t\}_{t \geq 0}$ (standardno) Brownovo gibanje, \(\tau\) čas ustavljanja glede na 
    \((\f_t)_{t \ge 0}\) in naj bo \(P[\tau \le \infty] = 1\).
    Potem je tudi proces:
    \[
    \hat{B} := \{B_{T+t} - B_T \mid t \geq 0\}
    \]
    (standardno) Brownovo gibanje in neodvisen od \(\f_\tau\).
    \end{izrek}
    
\bibliographystyle{plain}
\bibliography{knjiga}
\end{document}